The operation of pumps in WDNs has been optimized for several decades with
varying techniques and objectives \cite{mala-jetmarova_lost_2017}. A common theme of
existing approaches is the goal of reducing the energy cost of pumping. In
addition, several other constraints and objectives have been treated in the
literature, ranging from pressure bounds at consumer nodes
\cite{hajgato_deep_2020} over control of tank levels \cite{baran_multi-objective_2005} to robustness against leakage events
\cite{giustolisi_operational_2013}.
Traditional methods optimize a fixed pump schedule for a certain
period of time based on a computational model of the WDN and assumptions
on consumer demands \cite{ostfeld_ant_2008}. More recent methods aim to dynamically
adapt the schedule to the current state of the network
\cite{pei_real-time_2025}. RL is particularly promising for this
direction of research, as it models the pump scheduler as an agent in a dynamic
environment, deciding on the next action to apply based on current observations
of the network.
Previous approaches included tabular Q-learning \cite{candelieri_intelligent_2019}, usage of deep networks \cite{hajgato_deep_2020,ma_pump_2024} and scheduling of valves in large distribution networks under uncertainty \cite{belfadil_drl-epanet_2022}.
% A pioneering work by Candelieri \etal \cite{candelieri_intelligent_2019}
% applied tabular Q-Learning \cite{watkins_q-learning_1992}. Following
% approaches used deep networks as function approximators, enhancing the
% scalability to larger networks and more complex observations
% \cite{hajgato_deep_2020,ma_pump_2024}. Belfadil \etal successfully employed RL
% for scheduling of valves in large WDNs under uncertainty
% \cite{belfadil_drl-epanet_2022}.
However, most of the existing literature on
pump control assumes deterministic demands or knowledge of the demand pattern
by the agent. To address a more realistic use case, we evaluate the
performance of the agent under uncertainty. Our goal is to minimize the price
for pumping energy while making sure that the pressure at consumer nodes in
the network stays within acceptable bounds.

The work most closely related to ours is a very recent publication by Pei
\etal \cite{pei_real-time_2025}. The authors optimize an objective similar to
ours taking demand uncertainty into account. The
observations they provide to the agent differ throughout their experiments: In
one setup they assume very limited information comprised only of tank levels
and pump status, while in another they assume knowledge of demand forecasts
for every node in the network. We argue that the former underestimates
available information as knowledge from sensors distributed across the network
is not included while the latter might be an overestimation as demand information may not be available or reliable in the real world. %in cases where
%demand forecasts are only available for a subset of nodes, or the forecasting
%model is not reliable.
In order to further improve the practical use of RL for
pump scheduling, we conduct an in-depth analysis of the benefit of different observation types and different amounts of pressure sensors.
